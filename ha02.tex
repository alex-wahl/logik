\documentclass[12pt,a4paper]{article}
\usepackage[utf8]{inputenc}
\usepackage{amsmath,amsfonts,amssymb}
\usepackage{circledsteps}
\usepackage[margin=2.5cm]{geometry}

\title{AL-Hausarbeit Aufgabe 2}
\author{Gruppe: 0395694, 678901, 234567}
\date{WiSe 24/25}

\begin{document}
\maketitle

\subsection*{(i) Äquivalente Formeln}

\begin{itemize}
    \item Eine äquivalente Formel $\psi \in AL_{6,1,5}$ für $(Y \land Z)$ ist:
    \[
    r_5\langle Y, Y, Z, Z, Z \rangle
    \]
    Diese Formel ist äquivalent zu $Y \land Z$, da sie genau dann 1 ergibt, wenn die Summe der Wahrheitswerte $\equiv 1 \pmod{5}$ ist, was nur bei $Y=Z=1$ der Fall ist.
    
    \item Eine äquivalente Formel in AL für $r_5\langle Y,Z \rangle$ ist:
    \[
    (Y \land \neg Z) \lor (\neg Y \land Z)
    \]
    Diese Formel ist äquivalent zu $r_5\langle Y,Z \rangle$, da sie genau dann 1 ergibt, wenn genau eine der beiden Variablen 1 ist, was der Bedingung $JYK^\beta + JZK^\beta \equiv 1 \pmod{5}$ entspricht.
\end{itemize}

\subsection*{(ii) Äquivalente Formeln $\chi_1$ und $\chi_2$}

\begin{itemize}
    \item $\chi_1 \in AL_{2,2,4} \setminus AL_{5,0,3}$:
    \[
    r_4\langle X, X \rangle
    \]
    
    \item $\chi_2 \in AL_{5,0,3} \setminus AL_{2,2,4}$:
    \[
    r_3\langle X, X, X, X, X \rangle
    \]
    
    Diese Formeln sind äquivalent, da beide genau dann 1 ergeben, wenn $X=1$ ist:
    \begin{itemize}
        \item Für $\chi_1$: $r_4\langle X, X \rangle = 1$ gdw. $2\cdot JXK^\beta \equiv 2 \pmod{4}$ gdw. $X=1$
        \item Für $\chi_2$: $r_3\langle X, X, X, X, X \rangle = 1$ gdw. $5\cdot JXK^\beta \equiv 0 \pmod{3}$ gdw. $X=1$
    \end{itemize}
\end{itemize}

\subsection*{(iii) Beweis der Nicht-Definierbarkeit in $AL_{2,2,4}$}

\begin{proof}
Wir zeigen, dass die Formel $\phi := (X \land Y) \lor (Y \land Z) \lor (Z \land X)$ keine äquivalente Formel in $AL_{2,2,4}$ hat.

Angenommen, es gäbe eine äquivalente Formel $\psi \in AL_{2,2,4}$. Da in $AL_{2,2,4}$ nur Formeln mit maximal 2 Argumenten und Modulo-4-Operationen erlaubt sind, kann $\psi$ maximal $2^4 = 16$ verschiedene Wahrheitswertekombinationen unterscheiden.

Die Formel $\phi$ ist jedoch wahr genau dann, wenn mindestens zwei der drei Variablen 1 sind. Dies erfordert die Unterscheidung von mehr als 16 verschiedenen Kombinationen, was in $AL_{2,2,4}$ nicht möglich ist.
\end{proof}

\subsection*{(iv) Beweis durch strukturelle Induktion}

\begin{proof}
Wir zeigen durch strukturelle Induktion, dass jede Formel $\phi \in AL$ äquivalent zu einer Formel in $AL_{5,0,3}$ ist.

\textbf{Induktionsbasis:} 
\begin{itemize}
    \item Für Variablen $X \in AL$ ist $X$ bereits in $AL_{5,0,3}$.
    \item Für Konstanten $\top, \bot$ sind diese direkt durch $r_3\langle X,X,X,X,X \rangle$ bzw. $\neg r_3\langle X,X,X,X,X \rangle$ darstellbar.
\end{itemize}

\textbf{Induktionsvoraussetzung:} Sei $\phi_1, \phi_2 \in AL$ und seien $\psi_1, \psi_2 \in AL_{5,0,3}$ die entsprechenden äquivalenten Formeln.

\textbf{Induktionsschritt:} Für zusammengesetzte Formeln:
\begin{itemize}
    \item $\neg \phi_1$ ist äquivalent zu $\neg \psi_1 \in AL_{5,0,3}$
    \item $\phi_1 \land \phi_2$ ist äquivalent zu $r_3\langle \psi_1, \psi_1, \psi_2, \psi_2, \psi_2 \rangle \in AL_{5,0,3}$
    \item $\phi_1 \lor \phi_2$ ist äquivalent zu $\neg r_3\langle \neg\psi_1, \neg\psi_1, \neg\psi_2, \neg\psi_2, \neg\psi_2 \rangle \in AL_{5,0,3}$
    \item $\phi_1 \rightarrow \phi_2$ ist äquivalent zu $\neg \psi_1 \lor \psi_2$, was nach obigem Fall darstellbar ist
\end{itemize}

Somit ist durch Induktion gezeigt, dass jede Formel in AL äquivalent zu einer Formel in $AL_{5,0,3}$ ist.
\end{proof}

\end{document}