
\documentclass[12pt,a4paper]{article}
\usepackage[utf8]{inputenc}
\usepackage{amsmath,amsfonts,amssymb}
\usepackage{circledsteps}

\title{AL-Hausarbeit Aufgabe X}
\author{Gruppe: 0395694, 678901, 234567}
\date{WiSe 24/25}

\begin{document}
\maketitle
\section*{Aufgabe 1}

\section*{Hausaufgabe 2}

\subsection*{(i) Äquivalente Formeln für $(Y \land Z)$ und $r_5 \langle Y, Z \rangle$}
\begin{itemize}
    \item Eine äquivalente Formel $\psi \in AL_{6,1,5}$ für $(Y \land Z)$ ist:
    \[
    r_5 \langle Y, Z, \top, \top, \bot \rangle
    \]
    Diese Formel repräsentiert die Konjunktion, da $r_5$ die Modulo-Semantik nutzt, um zu garantieren, dass nur die Belegungen $(1, 1)$ wahr sind.
    
    \item Eine äquivalente Formel $\psi \in AL$ für $r_5 \langle Y, Z \rangle \in AL_{6,1,5}$ ist:
    \[
    (Y \lor Z) \land \neg (Y \land Z)
    \]
    Diese Formel entspricht der Modulo-Definition von $r_5$ für die Semantik, in der der Rest $1$ erfüllt sein muss.
\end{itemize}

\subsection*{(ii) Formeln $\chi_1$ und $\chi_2$ und ihre Äquivalenz}
\begin{itemize}
    \item Eine Formel $\chi_1 \in AL_{2,2,4} \setminus AL_{5,0,3}$ ist:
    \[
    r_4 \langle X, Y \rangle
    \]
    Diese Formel ist in $AL_{2,2,4}$ definiert, da sie modulo $4$ arbeitet, was in $AL_{5,0,3}$ nicht erlaubt ist.

    \item Eine Formel $\chi_2 \in AL_{5,0,3} \setminus AL_{2,2,4}$ ist:
    \[
    r_3 \langle X, Y, Z \rangle
    \]
    Diese Formel ist in $AL_{5,0,3}$ definiert, da sie auf Modulo $3$ basiert, was in $AL_{2,2,4}$ nicht zulässig ist.

    \item Die Äquivalenz kann durch die Semantik der jeweiligen Modulo-Operationen gezeigt werden: Beide Formeln bewirken eine spezifische Auswahl der Werte basierend auf der Restklassenarithmetik, jedoch in unterschiedlichen Systemen (Modulo $4$ vs. Modulo $3$).
\end{itemize}

\subsection*{(iii) Beweis, dass nicht jede Formel in $AL$ äquivalent zu einer in $AL_{2,2,4}$ ist}
\begin{proof}
Angenommen, es gäbe für jede Formel $\phi \in AL$ eine äquivalente Formel $\psi \in AL_{2,2,4}$. Nehmen wir eine Formel $\phi = r_5 \langle X, Y \rangle$. Diese Formel nutzt Modulo $5$, welches nicht in $AL_{2,2,4}$ unterstützt wird (nur Modulo $4$ ist erlaubt). Da die Modulo-Arithmetik nicht äquivalent dargestellt werden kann, ist $\phi$ nicht durch eine Formel in $AL_{2,2,4}$ ausdrückbar.
\end{proof}

\subsection*{(iv) Beweis durch strukturelle Induktion für $AL_{5,0,3}$}
\begin{proof}
Wir zeigen, dass jede Formel $\phi \in AL$ äquivalent zu einer Formel in $AL_{5,0,3}$ ist, indem wir strukturelle Induktion auf die Syntax von $AL$ anwenden.

\textbf{Induktionsanfang:} Für atomare Formeln $X \in AL$ ist $X$ direkt in $AL_{5,0,3}$ enthalten.

\textbf{Induktionsannahme:} Sei $\phi_1, \phi_2 \in AL$ und seien sie äquivalent zu Formeln $\psi_1, \psi_2 \in AL_{5,0,3}$.

\textbf{Induktionsschritt:} Für zusammengesetzte Formeln gilt:
\begin{itemize}
    \item $\neg \phi_1$: Da $\phi_1 \in AL_{5,0,3}$, ist auch $\neg \phi_1 \in AL_{5,0,3}$.
    \item $(\phi_1 \land \phi_2)$: Da $\phi_1, \phi_2 \in AL_{5,0,3}$, ist auch $(\phi_1 \land \phi_2) \in AL_{5,0,3}$.
    \item $r_3 \langle \phi_1, \phi_2 \rangle$: Diese Formel ist in $AL_{5,0,3}$ durch Definition der Modulo-Semantik enthalten.
\end{itemize}
Somit ist jede Formel in $AL$ äquivalent zu einer Formel in $AL_{5,0,3}$.
\end{proof}

\end{document}
