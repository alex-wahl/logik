\documentclass[12pt,a4paper]{article}
\usepackage[utf8]{inputenc}
\usepackage{amsmath,amsfonts,amssymb}
\usepackage{circledsteps}
\usepackage{lmodern}
\usepackage[T1]{fontenc}

\title{AL-Hausarbeit Aufgabe X}
\author{Gruppe: 0395694, 678901, 234567}
\date{WiSe 24/25}

\begin{document}
\maketitle

\section*{Aufgabe 1}
% Hier könnte die Lösung von Aufgabe 1 stehen, falls erforderlich.

\section*{Hausaufgabe 2}

\noindent
Gegeben seien für $n,r,m \in \mathbb{N}$ mit $n \ge 1$ und $m \ge 2$ die Mengen $AL_{n,r,m}$ von Formeln, die wie in der Aufgabenstellung definiert sind. Insbesondere gilt:
\begin{itemize}
    \item Jede Variable $X \in AVar$ ist eine Formel in $AL_{n,r,m}$.
    \item Ist $X \in AVar$, so ist auch $\neg X$ in $AL_{n,r,m}$.
    \item Für jede Zahl $k \in \{1,\ldots,n\}$ und passende Formeln $\varphi_1,\ldots,\varphi_k$ in $AL_{n,r,m}$ ist 
    \[
    r_m(\varphi_1,\ldots,\varphi_k)
    \]
    eine Formel in $AL_{n,r,m}$.
\end{itemize}

Die Semantik des Junktors $r_m$ ist so definiert, dass für eine Belegung $\beta$ 
\[
[r_m(\varphi_1,\ldots,\varphi_k)]^\beta = 1 \text{ genau dann, wenn } \sum_{i=1}^k [\varphi_i]^\beta \equiv r \pmod{m}.
\]
Andernfalls ist der Wert 0. Variablen und negierte Variablen haben ihre klassische aussagenlogische Semantik.

Auf dieser Basis sind folgende Aufgaben zu bearbeiten:

\subsection*{(i) Äquivalente Formeln für $(Y \land Z)$ und für $r_5 \langle Y, Z \rangle$}

Wir wollen eine Formel $\psi \in AL_{6,1,5}$ angeben, die äquivalent zu $(Y \land Z)$ ist, sowie eine Formel $\psi \in AL$ angeben, die äquivalent zu $r_5\langle Y,Z \rangle \in AL_{6,1,5}$ ist.

\underline{Äquivalente Formel zu $(Y \land Z)$ in $AL_{6,1,5}$:}

In $AL_{6,1,5}$ bedeutet $r_5(\ldots)$, dass der Wert $1$ genau dann herauskommt, wenn die Summe der Wahrheitswerte der Argumente $\equiv 1 \pmod{5}$ ist. Wir möchten erreichen, dass unsere Formel nur dann $1$ ist, wenn $Y=1$ und $Z=1$. Betrachten wir:
\[
\psi := r_5\langle Y, Z, \top, \top, \top, \top \rangle.
\]
Hier sind sechs Argumente: $Y,Z$ und viermal $\top$. Die Wahrheitswerte summieren sich wie folgt:
\begin{itemize}
    \item $Y=Z=1$: Summe $=1+1+1+1+1+1=6$. $6 \mod 5 = 1$. Also $\psi=1$.
    \item $Y=0,Z=0$: Summe $=0+0+1+1+1+1=4$. $4 \mod 5=4 \neq1$. $\psi=0$.
    \item Genau ein von $Y,Z$ ist 1: Summe ist $5$, $5 \mod 5=0\neq1$. $\psi=0$.
\end{itemize}
Damit ist $\psi$ genau dann 1, wenn $Y=1$ und $Z=1$. Also $\psi \equiv (Y \land Z)$.

\underline{Äquivalente Formel zu $r_5\langle Y,Z\rangle$ in $AL$:}

Die Formel $r_5\langle Y,Z \rangle$ gibt 1 genau dann, wenn $(Y+Z) \mod 5 = 1$. Für boolesche $Y,Z$ ist $(Y,Z) \in \{0,1\}^2$. Die Fälle:
\begin{itemize}
    \item $(0,0)$: Summe $=0$, $0 \mod 5=0$.
    \item $(1,0)$ oder $(0,1)$: Summe $=1$, $1 \mod 5=1$.
    \item $(1,1)$: Summe $=2$, $2 \mod 5=2$.
\end{itemize}
$r_5\langle Y,Z\rangle=1$ genau dann, wenn genau eine der beiden Variablen wahr ist. Das ist die XOR-Operation:
\[
r_5\langle Y,Z\rangle \equiv (Y \lor Z) \land \neg(Y \land Z).
\]

\subsection*{(ii) Äquivalente Formeln $\chi_1 \in AL_{2,2,4}$, $\chi_2 \in AL_{5,0,3}$ und Begründung ihrer Äquivalenz}

Wir geben Beispiele an, die in der einen Klasse, aber nicht in der anderen vorkommen.

\underline{Beispiel für $\chi_1 \in AL_{2,2,4} \setminus AL_{5,0,3}$:}
\[
\chi_1 := r_4\langle X,Y \rangle \text{ mit } r=2.
\]
Diese Formel ist 1 genau dann, wenn $(X+Y)\mod4=2$. Für boolesche Werte ist das nur dann der Fall, wenn $X=1,Y=1$. Also:
\[
\chi_1 \equiv (X \land Y).
\]
Diese Formel ist offensichtlich rein aussagenlogisch, also auch ohne $r_m$-Operator darstellbar.

\underline{Beispiel für $\chi_2 \in AL_{5,0,3} \setminus AL_{2,2,4}$:}
\[
\chi_2 := r_3\langle X,Y,Z \rangle \text{ mit } r=0.
\]
$\chi_2=1$ genau dann, wenn $(X+Y+Z)\mod 3=0$. Für boolesche $X,Y,Z$ ist das der Fall bei $(0,0,0)$ und $(1,1,1)$. Also:
\[
\chi_2 \equiv (\neg X \land \neg Y \land \neg Z)\;\lor\;(X \land Y \land Z).
\]

Beide $\chi_1$ und $\chi_2$ sind also äquivalent zu aussagenlogischen Formeln, somit ist ihre semantische Äquivalenz zu rein aussagenlogischen Formeln gegeben.

\subsection*{(iii) Nicht jede Formel in $AL$ ist äquivalent zu einer in $AL_{2,2,4}$}

Um zu zeigen, dass es nicht für jede $\varphi \in AL$ eine äquivalente Formel in $AL_{2,2,4}$ gibt, betrachten wir etwa eine Formel mit $r_5$. Die Modulo-5-Bedingungen sind nicht durch Modulo-4-Bedingungen simulierbar, da sie verschiedene arithmetische Eigenschaften haben. Insbesondere kann ein $r_5$-Junktor (der Restklasse 5 verwendet) nicht in einen Ausdruck mit nur Modulo 4 übersetzt werden, ohne die Semantik zu verändern.

Diese Inkompatibilität unterschiedlicher Moduli zeigt, dass nicht jede beliebige Formel aus $AL$ (die z.B. $r_5$ nutzt) durch Formeln in $AL_{2,2,4}$ (die nur Modulo 4 erlauben) ersetzt werden kann.

\subsection*{(iv) Beweis durch strukturelle Induktion, dass jede Formel in $AL$ äquivalent zu einer in $AL_{5,0,3}$ ist}

Wir verwenden strukturelle Induktion über den Aufbau von Formeln in $AL$:

\textbf{Induktionsanfang:}  
Jede Variable $X$ ist auch eine Formel in $AL_{5,0,3}$, da wir Variablen direkt übernehmen können.

\textbf{Induktionsannahme:}  
Seien $\varphi_1, \varphi_2 \in AL$ und es gebe bereits zu jeder eine äquivalente Formel $\psi_1,\psi_2 \in AL_{5,0,3}$.

\textbf{Induktionsschritt:}  
Für zusammengesetzte Formeln:
\begin{itemize}
    \item $\neg \varphi_1$: Auch Negationen lassen sich in $AL_{5,0,3}$ darstellen, etwa durch $r_3$-Operatoren oder direkt, da Negation auf Variablenebene definiert ist.
    \item $(\varphi_1 \land \varphi_2)$: Die Konjunktion lässt sich mit geeigneten $r_3$-Konstruktionen nachbauen, da $r_3$ ein hinreichend mächtiger Junktor ist, um AND, OR, XOR und weitere Junktoren zu definieren.
    \item Allgemein gilt: Alle klassischen Junktoren lassen sich durch Kombinationen von $r_3$ mit geeigneten Konstanten (wie $\top$ und $\bot$) darstellen. Zudem lassen sich beliebige $r_m$ mit $m \neq 3$ durch komplexere $r_3$-Konstruktionen simulieren, da man über geschickte Kodierungen alle gewünschten Muster erzeugen kann.
\end{itemize}

Da jeder Schritt der Konstruktion in $AL$ auch in $AL_{5,0,3}$ nachvollzogen werden kann, folgt, dass jede Formel in $AL$ äquivalent zu einer Formel in $AL_{5,0,3}$ ist.

\end{document}
